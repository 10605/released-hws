\documentclass{exam}

%------------------------ packages ------------------------%
\usepackage{amsmath,amsfonts,amsthm,amssymb,amsopn,bm}
\usepackage{pythontex}
\usepackage{fullpage}
\usepackage{graphicx}
\usepackage{fullpage}
\usepackage[paper=letterpaper,margin=1in,includeheadfoot,footskip=0.25in,headsep=0.25in]{geometry}
\usepackage{url}
\usepackage[usenames,dvipsnames]{color}
\usepackage[pdfborder={0 0 1},colorlinks=true,citecolor=black,plainpages=false]{hyperref}
\usepackage{multirow}
\usepackage[english]{babel}
\usepackage{pdfpages,bbm}
\usepackage{enumitem}
\usepackage{todonotes}

%------------------------ math ------------------------%
\newcommand{\R}{\mathbb{R}} % real domain
\newcommand{\Rset}{\mathbb{R}} % real domain
\newcommand{\argmin}{\operatorname{argmin}}
\newcommand{\argmax}{\operatorname{argmax}}
\newcommand{\xv}{\mathbf{x}}
\newcommand{\av}{\mathbf{a}}
\newcommand{\bv}{\mathbf{b}}
\newcommand{\yv}{\mathbf{y}}
\newcommand{\wv}{\mathbf{w}}
\newcommand{\uv}{\mathbf{u}}
\newcommand{\zv}{\mathbf{z}}
\newcommand{\Xv}{\mathbf{X}}
\newcommand{\W}{\mathbf{W}}
\newcommand{\K}{\mathbf{K}}
\newcommand{\M}{\mathbf{M}}
\newcommand{\C}{\mathbf{C}}
\newcommand{\B}{\mathbf{B}}
\newcommand{\X}{\mathbf{X}}
\newcommand{\I}{\mathbf{I}}
\newcommand{\Proj}{\mathbf{P}}
\newcommand{\Y}{\mathbf{Y}}
\newcommand{\U}{\mathbf{U}}
\renewcommand{\L}{\mathbf{L}}
\newcommand{\y}{\mathbf{y}}
\newcommand{\x}{\mathbf{x}}
\newcommand{\w}{\mathbf{w}}
\newcommand{\mD}{\mathbf{D}}
\newcommand{\A}{\mathbf{A}}
\newcommand{\zero}{\mathbf{0}}
\newcommand{\norm}[1]{\left\lVert#1\right\rVert}
\DeclareMathOperator{\rank}{rank}
\newcommand{\E}{\mathbb{E}}
\newcommand{\grade}[1]{\small\textcolor{magenta}{\emph{[#1 points]}} \normalsize}



\begin{document}

\title{Homework 3 Written Assignment}
\author{\Large \bf 10-605/10-805: Machine Learning with Large Datasets}
\date{{\bf Due Thursday, October 8th at 1:30:00 PM Eastern Time}}
\maketitle

\paragraph{Instructions:} Submit your solutions via Gradescope, \textit{with your solution to each subproblem on a separate page}, i.e., following the template below.  Note that Homework 3 consists of two parts: this written assignment, and a programming assignment. The written part is worth \textbf{60\%} of your total HW3 grade. The programming part makes up the remaining 40\%.

\paragraph{Submitting via Gradescope:} When submitting on Gradescope, you must assign pages to each question correctly (it
prompts you to do this after submitting your work). This significantly streamlines the
grading process for the course staff.  Failure to do this may result in a score of 0 for any questions
that you didn't correctly assign pages to. It is also your responsibility to make sure that your scan/submission is legible so that we can grade it.

\paragraph{Collaboration:}  While you may talk with others about the homework, you must each turn in your own solution and it should be your own work. \text{Please list the names of any collaborators below.}

\paragraph{Collaborators:}


\newpage



\section{Count-Min Sketch Basics (15 pts)} 


\begin{enumerate}[label=(\alph*)]

\item Let's use count-min sketch to estimate the counts of a stream of 5 items: $x_1, x_2, x_3, x_4, x_5$, using 3 hash functions: $h_1, h_2, h_3$, each with 3 buckets. The hash functions have the following collision matrix: 

\begin{center}
\begin{tabular}{ |p{3cm}||p{3cm}|p{3cm}|p{3cm}|  }
 \hline
 \multicolumn{4}{|c|}{collision matrix} \\
 \hline
 $h_1$   & $x_1, x_2$    &$x_4, x_5$ &   $x_3$ \\
 \hline
 $h_2$&   $x_2, x_3$  & $x_1, x_4$   & $x_5$ \\
 \hline
$h_3$ & $x_4$ & $x_1, x_5$ &  $x_2, x_3$ \\
 \hline
\end{tabular}
\end{center} \hspace{0pt}

Suppose you see the following stream of data: $\{x_1, x_2, x_5, x_3, x_3, x_4, x_4\}$.

\begin{enumerate}[label=\roman*.]
    \item \grade{2} What is the value of $\hat{c}_5$, the approximate count of item $x_5$, based on this sketch? 
    
    \vspace{2em} 
    \item \grade{2} What is the value of $\hat{c}_1$, the approximate count of item $x_1$, based on this sketch?
    
    \vspace{2em} 
    \item \grade{4} Let $c_s$ be the true count of item $x_s$. What is the average absolute error of approximation for all items, i.e. $\frac{1}{5}\sum_{s=1}^5|\hat{c}_s-c_s|$?
    
    \vspace{3em}
    
\end{enumerate}


\item \grade{7} In your own words, explain the approximation guarantees we provided in lecture for count-min sketch. What do these results tell you about how to configure count-min sketch (i.e., how to determine the number of hash functions and buckets) to achieve a certain approximation quality?

\end{enumerate}





\newpage
\section{An Unbiased Sketch (30 pts)}
  The count-min sketch presented in lecture is {biased}, as the estimated count $\hat{c}_s$ for item $s \in \{1,\ldots,k\}$ is always
  higher than (or equal to) the true count $c_s$. In this problem,
  we will explore an \textit{unbiased} sketch.  \\

\noindent Let
  $g$ be a hash function chosen from a family $G$ of independent
  hashes, such that $g$ maps each item $s$ to either $+1$ or $-1$ with
  equal probability:
 \[
 P(g(s) = +1) = P(g(s) = -1) = 1/2. 
 \]
 Let's start with a simple version of the sketch, which uses one additional hash function, $h$, which maps each item $s$ to one of $m$ buckets. 
Assume that $g$ and $h$ are independent. When we observe an item $s$ in the
sequence, we simply update the sketch counter, $C$, as:
\[
C[h(s)] =C[h(s)] + g(s).
\]
Then, if we would like to predict the count for item $s$,
we return:
\[
\hat{c}_s = C[h(s)] \  g(s).
\]
Given this sketch, please answer the following questions:

\begin{itemize}

\item[{a)}] \grade{8} First, show that this simple sketch is unbiased, i.e., $\E[\hat{c}_s] = c_s$. (\textit{Hint: Find an expression for $\hat{c}_s$ in terms of $g(s)$ and $c_s$, and use linearity of expectation.})  \\

\newpage
\item[{b)}] \grade{8} Let's understand how much the estimates vary. Prove that: $Var(\hat{c}_s) = \frac{1}{m}\sum_{j \in \{1, \dots, k\}, j \neq s} c_j^2$, where $m$ is the number of buckets for the hash function $h$.



\newpage
\item[{c)}] \grade{6} We will now bound the probability of getting a bad
  estimate $\hat{c}_s$, i.e., one that differs substantially from the true count $c_s$. Show that:

\[
P(|\hat{c}_s - c_s| \ge \epsilon \|\mathbf{c}_{-s}\|_2) \le \frac{1}{\epsilon^2 m}\, ,
\]
where $\mathbf{c}_{-s}$ is a vector of length $(k - 1)$ containing the counts for all items except $s$. \textit{(Hint: Consider Chebyshev's inequality.)} \\


\newpage
\item[{d)}] Similar to count-min sketch, we can improve the results from (c) by using $r$ hash functions for mapping the items to buckets $(h_1, \dots, h_r)$, and $r$ hash functions for assigning signs $(g_1, \dots g_r)$. In particular, setting $m > \frac{3}{\epsilon^2}$ and $r > (log(1/\delta))$, we can guarantee: 
$$
P(|\hat{c}_s - c_s| \ge \epsilon \|\mathbf{c}_{-s}\|_2) \le \delta\, .
$$
Note that while this result appears similar to the one we had for count-min sketch, it requires more buckets, $m$. Recall that for count-min sketch with $m > \frac{2}{\epsilon}$ and $r > (log(1/\delta))$, we had the bound:
$$
P(|\hat{c}_s - c_s| \ge \epsilon \|\mathbf{c}\|_1) \le \delta \, .
$$
With these results in mind, answer the questions below. 
\begin{enumerate}[label=\roman*.]
\item \grade{4} Explain a scenario where count-min sketch would be preferable to the unbiased sketch we derived (using the above bounds). \\

\vspace{8em}

\item \grade{4} Explain a scenario where the unbiased sketch we derived would be preferable to count-min sketch (using the above bounds). \\


\end{enumerate}

\end{itemize}

\newpage
\section{Locality-Sensitive Hashing (25 pts)}

In lecture, we saw that LSH can be used for applications such as nearest neighbors classification and clustering. We mentioned that if two points $\xv$ and $\yv$ have small \textit{cosine distance}, then the resulting bit vectors formed by using LSH, $\xv'$ and $\yv'$, will have small \textit{Hamming distance}. In this problem, we will explore this result in more detail. \\

\noindent First, let's recall the definitions of cosine similarity, cosine distance, and Hamming distance.\\

\noindent \textit{Cosine similarity} measures the cosine of the angle between two vectors $\xv, \yv \in \R^k$, and can be calculated as:
$$ s_{cos}(\xv,\yv) = \cos(\theta(\xv,\yv)) =  \frac{\xv \cdot \yv}{\|\xv\|_2 \|\yv\|_2} \, $$

\noindent \textit{Cosine distance} is measured as $d_{cos}(\xv,\yv) = 1-s_{cos}(\xv,\yv).$ \\

\noindent \textit{Hamming distance} between two \textit{binary} (0-1) vectors $\xv', \yv'$ of length $k$ measures the number of differing bits, and is calculated as:
$$ d_{h} (\xv',\yv') = \sum_{i=1}^k |\xv'_i - \yv'_i| \, $$
\vspace{1em}

\begin{enumerate}[label=(\alph*)]
    \item\grade{2} Define $\av = [0,0,1,1,0]$, $\bv = [0,1,0,0,1]$. What's the \textit{cosine similarity} between $\av$ and $\bv$?
    

    \vspace{5em}
    \item\grade{2} Define $\av = [0,0,1,1,0]$, $\bv = [0,1,0,0,1]$. What's the \textit{Hamming distance} between $\av$ and $\bv$?
    

    \vspace{1em}
    
    
    \newpage
    \item \grade{6} For a random hyperplane, $\zv$, define the following hash function: 
    $$
h_{z}(\xv) =
\begin{cases}
1 \quad \text{if} \quad \zv \cdot \xv \ge 0  \\
0 \quad \text{if} \quad \zv \cdot \xv < 0
\end{cases}
$$
For two vectors $\xv,\yv \in \R^k$, prove that:
$$P[h_z(\xv) = h_z(\yv)] = 1 - \frac{\theta(\xv,\yv)}{\pi} \, ,$$
where $\theta(\xv,\yv)$ is the angle between vectors $\xv, \yv$ measured in radians. 

To prove this, use the fact that $P[\zv \cdot \xv \ge 0, \zv \cdot \yv < 0 ] = \frac{\theta(\xv,\yv)}{2\pi}$, i.e., the probability that a random hyperplane separates two vectors is proportional to the angle between them. 

\newpage
\item \grade{6} In LSH, we estimate $P[h_z(\xv)=h_z(\yv)]$ by selecting $b$ random hyperplanes, and calculating bit vectors $\xv'_i = h_i(\xv)$ and $\yv'_i = h_i(\yv)$ for $i=1,\dots,b$. Suppose that after running this procedure, we compute the Hamming distance and find: $d_h(\xv',\yv')=h$. Using this estimate and part (c), show that:
$$ d_{cos}(\xv,\yv) \approx 1 - \cos\left(\frac{h}{b} \pi\right)$$
Note that this equation shows that if $h=b$, i.e., the bit vectors have a large hamming distance, the cosine distance will also be large ($d_{cos}(\xv,\yv) = 2$). Similarly, for $h=0$, we will have $d_{cos}(\xv,\yv) = 0$.

    
    \newpage
    \item Let's go through a simple example using LSH. Suppose we have two vectors, $\xv = [3,4,5,6]$ and $\yv=[4,3,2,1]$.
    \begin{enumerate}[label=\roman*.]
        \item \grade{3} Use one random vector: $\zv_1 = [0.16, -0.26, -1.12, 0.01]$ to perform LSH. Compute the error of the resulting estimated cosine distance, i.e., $|1 - \cos\left(\frac{h}{b}\pi\right) - d_{cos}(\xv,\yv)|$. 
        
        \vspace{8em}
        \item \grade{3} Now let's try with more vectors. Use four random vectors: $\zv_1 = [0.16, -0.26, -1.12, 0.01]$, $\zv_2=[1.08,-1.14,0.8,-0.8]$, $\zv_3 = [1.94, 0.08, 1.83, 0.17]$, $\zv_4=[-0.88, -1.01, 1.36, -0.07]$ to perform LSH. Compute the error of the estimated cosine distance.
        
        \vspace{8em}
        \item \grade{3} In practice, it can be sufficient to consider random vectors only consisting of $\{-1,+1\}$ when performing LSH. Use the signs of the previous random vectors, i.e., $\zv_1 = [+1, -1, -1, +1]$, $\zv_2=[+1,-1,+1,-1]$, $\zv_3 = [+1,+1,+1,+1]$, $\zv_4=[-1,-1,+1,-1]$, to perform LSH. Compute the error of the estimated cosine distance.
        
    \end{enumerate}
\end{enumerate}

\end{document}