\documentclass{exam}

%------------------------ packages ------------------------%
\usepackage{amsmath,amsfonts,amsthm,amssymb,amsopn,bm}
\usepackage{pythontex}
\usepackage{fullpage}
\usepackage{graphicx}
\usepackage{fullpage}
\usepackage[paper=letterpaper,margin=1in,includeheadfoot,footskip=0.25in,headsep=0.25in]{geometry}
\usepackage{url}
\usepackage[usenames,dvipsnames]{color}
\usepackage[pdfborder={0 0 1},colorlinks=true,citecolor=black,plainpages=false]{hyperref}
%\usepackage{fancyhdr}
\usepackage{multirow}
\usepackage[english]{babel}
\usepackage{pdfpages,bbm}
\usepackage{enumitem}
\usepackage{todonotes}


%------------------------ math ------------------------%
\newcommand{\R}{\mathbb{R}} % real domain
\newcommand{\Rset}{\mathbb{R}} % real domain
\newcommand{\argmin}{\operatorname{argmin}}
\newcommand{\argmax}{\operatorname{argmax}}
\newcommand{\xv}{\mathbf{x}}
\newcommand{\wv}{\mathbf{w}}
\newcommand{\Xv}{\mathbf{X}}
\newcommand{\W}{\mathbf{W}}
\newcommand{\K}{\mathbf{K}}
\newcommand{\M}{\mathbf{M}}
\newcommand{\C}{\mathbf{C}}
\newcommand{\B}{\mathbf{B}}
\newcommand{\X}{\mathbf{X}}
\newcommand{\I}{\mathbf{I}}
\newcommand{\Proj}{\mathbf{P}}
\newcommand{\Y}{\mathbf{Y}}
\newcommand{\U}{\mathbf{U}}
\renewcommand{\L}{\mathbf{L}}
\newcommand{\y}{\mathbf{y}}
\newcommand{\x}{\mathbf{x}}
\newcommand{\w}{\mathbf{w}}
\newcommand{\mD}{\mathbf{D}}
\newcommand{\A}{\mathbf{A}}
\newcommand{\zero}{\mathbf{0}}
\newcommand{\norm}[1]{\left\lVert#1\right\rVert}
\DeclareMathOperator{\rank}{rank}


%------------------------ exam class macros ------------------------%
\checkboxchar{$\Box$}
\renewcommand{\checkboxeshook}{
  \settowidth{\leftmargin}{W.}
  \labelwidth\leftmargin\advance\labelwidth-\labelsep
}

\newcommand{\grade}[1]{\small\textcolor{magenta}{\emph{[#1 points]}} \normalsize}

\begin{document}

\title{Homework 2 Written Assignment}
\author{\Large \bf 10-605/10-805: Machine Learning with Large Datasets}
\date{{\bf Due Tuesday, September 29th at 1:30:00 PM Eastern Time}}
\maketitle

\noindent Submit your solutions via Gradescope, \textbf{with your solution to each subproblem on a separate page}, i.e., following the template below.  Note that Homework 2 consists of two parts: this written assignment, and a programming assignment. The written part is worth \textbf{30\%} of your total HW2 grade (programming part makes up the remaining 70\%).

\newpage

\section{Nystr\"{o}m Method (30 points)} Nystr\"{o}m method. Define the following block representation of a kernel matrix:
\vspace{-.5em}
\begin{equation*}
\K = 
\begin{bmatrix}
      \W & \K_{21}^\top\\
      \K_{21} & \K_{22}
\end{bmatrix}
\quad \text{and} \quad
\C = 
\begin{bmatrix}
\W \\
\K_{21}
\end{bmatrix}.
\end{equation*}
The Nystr\"{o}m method uses $\W \in \Rset^{l \times l}$, $\C \in \Rset^{m \times
l}$ and $\K \in \Rset^{m\times m}$ to generate the approximation $\widetilde \K = \C \W^\dagger \C^\top \approx \K$.

\begin{enumerate}[label=(\alph*)]

    \item \grade{5} Show that $\W$ is symmetric positive semi-definite (SPSD) and that $\norm{\K-\widetilde \K}_F =
\norm{\K_{22} - \K_{21}\W^\dagger \K_{21}^\top}_F$, where $\norm{.}_F$ is the \href{https://en.wikipedia.org/wiki/Matrix_norm#Frobenius_norm}{Frobenius norm}.

    \newpage
    \item \grade{10} Let $\K = \X^\top \X$ for some $\X \in \Rset^{N \times m}$, and let
$\X'\in \Rset^{N \times l}$ be the first $l$ columns of $\X$. Show that
$\widetilde \K = \X^\top \Proj_{U_{X'}} \X$, where $\Proj_{U_{X'}}$ is the
orthogonal projection onto the span of the left singular vectors of $\X'$.

   \newpage    
   \item \grade{5} Is $\widetilde \K$ symmetric positive semi-definite (SPSD)?
   
   \newpage    
   \item \grade{5} If $\rank(\K) = \rank(\W) = r \ll m$, show that $\widetilde
\K = \K$. Note: this statement holds whenever $\rank(\K) =
\rank(\W)$, but is of interest mainly in the low-rank setting.

   \newpage    
   \item \grade{5} If $m = 20$M and $\K$ is a dense matrix, how much space is required to
store $\K$ if each entry is stored as a double? How much space is required by
the Nystr\"{o}m method if $l=10$K?

\end{enumerate}
\end{document}